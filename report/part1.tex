\section{Введение}
В рамках курса было необходимо разработать приложение, позволяющее 
продемонстрировать применение основных принципов разработки программного 
обеспечения.
В частности, в приложении необходимо было выделить следующие компоненты:
\begin{itemize}
\item слой бизнесс-логики;
\item слой хранения данных;
\item слой представления.
\end{itemize}
Было решено разработать приложение, основное назначение которого --- упростить процесс разработки программных проектов для какой-либо группы разработчиков.

В ней определены следующие объекты:
\begin{itemize}
	\item \textbf{Проект} У каждого проекта есть определенная команда разработчиков, тестировщиков и один менеджер. Также к проекту может быть привязан тимлидер. У проекта определены различные майлстоуны. К каждому проекту могут быть привязаны сообщения об ошибках.
	
	\item \textbf{Майлстоун} Одна из итераций цикла разработки проекта. Привязан к определенным датам. К майлстоунам привязаны определенные тикеты~(задания). Майлстоун имеет определенный статус: открыт, активен или закрыт. Майлстоун может быть закрыт только когда все его тикеты выполнены. В каждый момент времени у проекта может быть только один майлстоун.
	
	\item \textbf{Тикет} Определенное задание для разработчиков. Может быть выдано определенной группе разработчиков. Привязан к определенному проекту и майлстоуну. Имеет статус: новый, принятый, в процессе выполнения, выполнен.
	
	\item \textbf{Сообщение об ошибке} Отчет о найденной ошибке в проекте. Привязан к определенному проекту. Имеет статус: новый, исправленный, протестированный, закрытый.
\end{itemize}