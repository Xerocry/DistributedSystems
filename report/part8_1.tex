\section{Тестирование}
Было проведено ручное функциональное тестирование приложения. Были протестированы все бизнес-процессы в приложении, проверена обработка ошибочных ситуаций. Список проводимых тестов:
\begin{itemize}
	\item \textbf{Попытка добавить пользователя, который уже существует}. Окрываем окно регистрации нового пользователя, вводим данные пользователя. В поле \texttt{Login} указываем логин уже зарегистрированного пользователя. Нажимаем кнопку \texttt{Register}. В результате появится окно с сообщением <<User \texttt{login} already exists>>.
	
	\item \textbf{Регистрация нового пользователя}. Открыть окно регистрации пользователя. Ввести уникальные данные пользователя. Нажать кнопку \texttt{Register}. Регистрация должна пройти успешно. Должно отрыться окно входа в систему.
	
	\item \textbf{Попытка войти в систему с неправильным логином или паролем}. В окне входа в систему ввести неправильные данные пользователя (логин или пароль). Нажать кнопку \texttt{Sign In}. Должно появиться окно с сообщением <<Incorrect login or password>>.
	
	\item \textbf{Вход в систему}. В окне входа в систему ввести правильные данные пользователя. Нажать кнопку \texttt{Sign In}. Должно открыться окно указанного пользователя.
	
	\item \textbf{Создание проекта с дублирующимся именем}. В окне пользователя в строке добавления нового проекта ввести имя нового проекта, которое полностью совпадает с именем уже существующего проекта. Нажать кнопку \texttt{Create project}. Должно появиться окно ошибки с сообщением <<Project \texttt{name} already exists>>.
	
	\item \textbf{Создание нового проекта}. В окне пользователя в строке добавления нового проекта ввести уникальное имя нового проекта. Нажать кнопку \texttt{Create project}. В списке <<Managed projects>> должен появиться новый проект с указанным именем.
	
	\item \textbf{Установка несуществующего пользователя тимлидером}. Открыть окно нового проекта. Должна быть доступна кнопка \texttt{Set team leader}. Ввести несуществующий логин пользователя и нажать на кнопку. Должно появиться окно ошибки с сообщением <<User \texttt{login} not found>>.
	
	\item \textbf{Назначение тимлидера}. Ввести данные существующего пользователя и нажать на кнопку \texttt{Set team leader}. Операция должна успешно завершиться. В окне проекта возле пункта \texttt{Team leader} должно появиться имя указанного пользователя.
	
	\item \textbf{Добавление участников в проект}.
	\begin{itemize}
		\item В окне проекта под списком разрабочиков ввести логин несуществующего пользователя и нажать кнопку \texttt{Add developer}. Должно появиться окно ошибки с сообщением <<User \texttt{login} not found>>.
		\item В окне проекта под списком разрабочиков ввести логин тимлидера проекта и нажать кнопку \texttt{Add developer}. Должно появиться окно ошибки с сообщением <<User \texttt{login} already have a role \texttt{role} in project \texttt{name}>>.
		\item В окне проекта под списком разрабочиков ввести логин существующего пользователя и нажать кнопку \texttt{Add developer}. Операция должна завершиться успешно. В списке разработчиков должна появиться новая строчка с именем указанного пользователя.
		\item В окне проекта под списком тестировщиков ввести логин несуществующего пользователя и нажать кнопку \texttt{Add tester}. Должно появиться окно ошибки с сообщением <<User \texttt{login} not found>>.
		\item В окне проекта под списком тестировщиков ввести логин тимлидера проекта и нажать кнопку \texttt{Add tester}. Должно появиться окно ошибки с сообщением <<User \texttt{login} already have a role \texttt{role} in project \texttt{name}>>.
		\item В окне проекта под списком тестировщиков ввести логин существующего пользователя и нажать кнопку \texttt{Add tester}. Операция должна завершиться успешно. В списке тестировщиков должна появиться новая строчка с именем указанного пользователя.
	\end{itemize}
	
	\item \textbf{Добавление майлстоуна, у которого дата завершения раньше чем дата начала}. В окне проекта под списком майлстоунов ввести даты начала и завершения майлстоуна так, чтобы дата завершения была перед датой начала. Нажать \texttt{Add milestone}. Должно появиться окно ошибки с сообщением <<Can't create milestone with end date before start date>>.
	
	\item \textbf{Добавление майлстоуна, у которого дата завершения в прошлом}. В окне проекта под списком майлстоунов ввести даты начала и завершения майлстоуна так, чтобы дата завершения была в прошлом. Нажать \texttt{Add milestone}. Должно появиться окно ошибки с сообщением <<Can't create milestone with end date in the past>>.
	
	\item \textbf{Добавление майлстоуна}. В окне проекта под списком майлстоунов ввести корректные даты начала и завершения майлстоуна. Нажать \texttt{Add milestone}. В списке майлстоунов проекта должна появиться новая запись, соответствующая созданному майлстоуну. Статус появившегося майлстоуна --- <<OPENED>>.
	
	\item \textbf{Неправильная смена статуса майлстоуна}. Открыть окно нового майлстоуна. Попытаться изменить статус майлстоуна с <<OPENED>> на <<CLOSED>> нажав на кнопку \texttt{Close}. Должно появиться окно ошибки с сообщением <<Cannot change status from OPENED to CLOSED>>.
	
	\item \textbf{Добвление тикета}. В окне майлстоуна ввести описание нового тикета и нажать на кнопку \texttt{Add ticket}. В списке тикетов майлстоуна должна появиться новая строчка с описанием созданного тикета.
	
	\item \textbf{Активация майлстоуна}. В окне нового майлстоуна нажать кнопку \texttt{Activate}. Операция должна пройти успешно. Статус майлстоуна должен поменяться на <<ACTIVE>>. В окне майлстоуна справа от строчки \texttt{Activated date} должна появиться текущая дата.
	
	\item \textbf{Два активных майлстоуна одновременно}. В окне проекта создать новый майлстоун. Открыть окно нового майлстоуна и нажать кнопку \texttt{Activate}. Должно появиться окно ошибки с сообщением <<Attempting to activate milestone \texttt{id1}, when milestone \texttt{id2} is already active>>.
	
	\item \textbf{Закрыть майлстоун когда не все его тикеты закрыты}. Окрыть окно активного майлстоуна. Нажать кнопку \texttt{Close}. Должно появиться окно ошибки с сообщением <<Ticket \texttt{id} is not closed>>.
	
	\item \textbf{Добавить разработчика в тикет}. Открыть окно нового тикета. Под списком разработчиков вести логин разработчика, являющегося разработчиком в проекте. Нажать \texttt{Add assignee}. В списке разработчиков тикета должна появиться новая строчка с именем указанного пользователя.
	
	\item \textbf{Добавить неправильного пользователя в проект}. В окне тикета ввести логин пользователя, который существует в системе но никак не связан с проектом. Нажать \texttt{Add assignee}. Должно появиться окно ошибки с сообщением <<User \texttt{login} has no permission \texttt{permission} for project \texttt{name}>>.
	
	\item \textbf{Проверка добавления разработчика}. Выйти из аккаунта менеджера. Войти в всистему за пользователя, которого только что назначили разработчиком тикета. В главном окне пользователя быть одна запись в списке разрабатываемых тикетов --- созданный ранее тикет.
	
	\item \textbf{Поменять статус тикета на <<ACCEPTED>>}. Октрыть окно тикета. Нажать на кнопку \texttt{Set accepted}. Статус тикета должен поменяться. В списке комментариев тикета должна появиться соответствующая запись.
	
	\item \textbf{Поменять статус тикета на <<IN\_PROGRESS>>}. Октрыть окно тикета. Нажать на кнопку \texttt{Set in progress}. Статус тикета должен поменяться. В таблице комментариев тикета должна появиться соответствующая запись.
	
	\item \textbf{Поменять статус тикета на <<FINISHED>>}. Октрыть окно тикета. Нажать на кнопку \texttt{Set finished}. Статус тикета должен поменяться. В таблице комментариев тикета должна появиться соответствующая запись.
	
	\item \textbf{Закрыть тикет}. Выйти из системы. Войти в систему за пользователя \texttt{manager}. Открыть окно тикета. Нажать на кнопку \texttt{Set closed}. Статус тикета должен поменяться. В таблице комментариев тикета должна появиться соответствующая запись.
	
	\item \textbf{Закрыть майлстоун}. Открыть окно майлстоуна, к которому был привязан тикет. Нажать на кнопку \texttt{Close}. Операция должна завершиться успешно. Статус майлстоуна должен поменяться на <<CLOSED>>. В окне майлстоуна справа от строчки \texttt{Closing date} должна появиться текущая дата.
	
	\item \textbf{Добавить тикет в закрытый майлстоун}. Открыть окно закрытого майлстоуна. Ввести описание нового тикета и нажать на кнопку \texttt{Add ticket}. Должно появиться окно ошибки с сообщением <<Milestone \texttt{id} of project \texttt{name} is already closed>>.
	
	\item \textbf{Создать багрепорт}. Выйти из системы. Войти в систему за пользователя \texttt{teamleader}. Открыть окно проекта. Ввести описание ошибки и нажать на кнопку \texttt{Add report}. В списке ошибок должна появиться новая строчка, соответствующая созданному ошибок.
	
	\item \textbf{Принять багрепорт}. Выйти из системы. Войти в систему за пользователя \texttt{developer}. Открыть окно багрепорта. Поменять статус на <<ACCEPTED>>. Статус багрепорта должен поменяться. В таблице комментариев должен появиться соответствующий комментарий. На странице багрепорта справа от пункта \texttt{Developer} должно появиться имя текущего пользователя.
	
	\item \textbf{Принять принятый багрепорт}. Выйти из системы. Войти в систему за пользователя \texttt{teamleader}. Открыть окно багрепорта. Пoменять статус на <<ACCEPTED>>. Должно появиться окно ошибки с сообщением, что другой пользователь уже принял этот багрепорт.
	
	\item \textbf{Багрепорты разработчика}. Выйти из системы. Войти в систему за пользователя \texttt{developer}. В списке разрабатываемых багрепортов на странице пользователя должна появиться запись, соответствующая принятому багрепорту.
	
	\item \textbf{Исправить багрепорт}. Открыть окно багрепорта. Поменять статус на <<FIXED>>. Статус багрепорта должен поменяться. В таблице комментариев должен появиться соответствующий комментарий.
	
	\item \textbf{Переоткрыть багрепорт}. Выйти из системы. Войти в систему за пользователя \texttt{tester}. Открыть окно багрепорта. Поменять статус на <<OPENED>>. Статус багрепорта должен поменяться. В таблице комментариев должен появиться соответствующий комментарий.
	
	\item \textbf{Снова исправить багрепорт}. Выйти из системы. Войти в систему за пользователя \texttt{developer}. Открыть окно багрепорта. Поменять статус на <<FIXED>>. В появившемся окне ввести комментарий. Статус багрепорта должен поменяться. В таблице комментариев должен появиться соответствующий комментарий.
	
	\item \textbf{Закрыть багрепорт}. Выйти из системы. Войти в систему за пользователя \texttt{tester}. Открыть окно багрепорта. Поменять статус на <<CLOSED>>. В появившемся окне ввести комментарий. Статус багрепорта должен поменяться. В таблице комментариев должен появиться соответствующий комментарий.
\end{itemize}

Все описанные тесты успешно выполняются. Все полученные результаты совпажают с ожидаемыми. Функциональное тастирование позволило понять, что созданной приложение работает корректно и выполняет свои функции.