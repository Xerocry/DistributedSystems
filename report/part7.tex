\section{Слой хранения данных}
	В качестве СУБД была выбрана система MySQL. Для работы с базой данных был использован Java Persistance API, в частности его реализация в библиотеке Hibernate. В базе данных хранятся все сущности, описанные в пакете \texttt{entity}. Описание сущностей производится с помощью специальных аннотаций JPA. Пример сущностного класса \texttt{User} приведен в листинге~\ref{lst:userORM}.
	\begin{lstlisting}[style=crs_java, label={lst:userORM}, caption={Описание сущности}]
	@Entity
	@Table(name = "USERS")
	public class User {
	
	@Id
	@Column(name = "ID")
	@GeneratedValue
	private Long id;
	
	@Column(name = "LOGIN", unique = true, nullable = false)
	private String login;
	
	@Column(name = "NAME")
	private String name;
	
	@Column(name = "PASSWORD")
	private String password;
	
	@JsonIgnore
	@OneToMany(fetch = FetchType.EAGER, mappedBy = "owner")
	private List<Message> messages = new ArrayList<>();
	}
	\end{lstlisting}
	
	Также, для того чтобы Hibernate нашел все сущности, необходимо описать все сущностные классы в файле persistance.xml(листинг~\ref{lst:persistance}).
	\begin{lstlisting}[style=crs_xml, label={lst:persistance}, caption={Файл persistance.xml}]
	<?xml version="1.0" encoding="UTF-8" ?>
	<persistence xmlns="http://java.sun.com/xml/ns/persistence"
	xmlns:xsi="http://www.w3.org/2001/XMLSchema-instance"
	xsi:schemaLocation="http://java.sun.com/xml/ns/persistence
	http://java.sun.com/xml/ns/persistence/persistence_1_0.xsd" version="1.0">
	
	<persistence-unit name="my-pms">
	<class>com.kspt.pms.entity.BugReport</class>
	<class>com.kspt.pms.entity.Comment</class>
	<class>com.kspt.pms.entity.Message</class>
	<class>com.kspt.pms.entity.Milestone</class>
	<class>com.kspt.pms.entity.Project</class>
	<class>com.kspt.pms.entity.Ticket</class>
	<class>com.kspt.pms.entity.User</class>
	</persistence-unit>
	
	</persistence>
	\end{lstlisting}
	
	Для каждой сущности был реализован свой репозиторий, в котором определены методы для поиска и сохранения сущностей. Репозитории были реализованы с помощью библиотеки Spring Data. Данная библиотека позволяет описывать только интерфейс, и по названиям методов этого интерфейса умеет автоматически генерировать его реализацию. Это очень сильно упрощает работу с базой данных. Рассмотрим данные репозитории более подробно.
	\begin{itemize}
		\item \texttt{BugReportRepository}
		\begin{itemize}
			\item \texttt{Optional<BugReport> findById(Long id)} --- поиск по идентификатору.
			\item \texttt{Collection<BugReport> findByProjectName(String name)} --- поиск всех отчетов, принадлежащих проекту с заданным именем.
			\item \texttt{Collection<BugReport> findByCreatorLogin(String login)} --- поиск всех отчетов, созданных пользователем с заданным логином.
			\item \texttt{Collection<BugReport> findByDeveloperLogin(String login)} --- поиск всех отчетов, исправляемых пользователем с заданным логином.	
		\end{itemize}
		
		\item \texttt{CommentRepository}
		\begin{itemize}
			\item \texttt{Optional<Comment> findById(Long id)} --- поиск по идентификатору.
			\item \texttt{Collection<Comment> findByUserLogin(String login)} --- поиск всех комментариев, оставленных пользователем с заданным логином.
		\end{itemize}
		
		\item \texttt{MessageRepository}
		\begin{itemize}
			\item \texttt{Collection<Message> findByOwnerLogin(String login)} --- поиск всех уведомлений, принадлежащих пользователю с заданным логином.
		\end{itemize}
		
		\item \texttt{MilestoneRepository}
		\begin{itemize}
			\item \texttt{Optional<Milestone> findById(Long id)} --- поиск по идентификатору.
			\item \texttt{Collection<Milestone> findByProjectName(String name)} --- поиск всех майлстоуну, принадлежащих проекту с заданным именем.
		\end{itemize}
		
		\item \texttt{ProjectRepository}
		\begin{itemize}
			\item \texttt{Optional<Project> findByName(String name)} --- поиск по имени.
			\item \texttt{Collection<Project> findByManagerLogin(String login)} --- поиск по менеджеру.
			\item \texttt{Collection<Project> findByTeamLeaderLogin(String login)} --- поиск по тимлидеру.
			\item \texttt{Collection<Project> findByDevelopersContaining(User user)} --- поиск по разработчику.
			\item \texttt{Collection<Project> findByTestersContaining(User user)} --- поиск по тестировщику.
		\end{itemize}
		
		\item \texttt{TicketRepository}
		\begin{itemize}
			\item \texttt{Optional<Ticket> findById(Long id)} --- поиск по идентификатору.
			\item \texttt{Collection<Ticket> findByMilestoneId(Long id)} --- поиск майлстоуну.
			\item \texttt{Collection<Ticket> findByAssigneesContaining(User user)} --- поиск по исполнителю.
			\item \texttt{Collection<Ticket> findByCreatorLogin(String login)} --- поиск по создателю.
		\end{itemize}
		
		\item \texttt{UserRepository}
		\begin{itemize}
			\item \texttt{Optional<User> findByLogin(String login)} --- поиск по логину.
		\end{itemize}
	\end{itemize}
	
	Для того, чтобы все эти библиотеки правильно проинициализировались, необходимо в классе конфигурации определить источник данных~(листинг~\ref{lst:config}).
	\begin{lstlisting}[style=crs_java, label={lst:config}, caption={Конфигурация источника данных}]
	@Bean
	public DataSource dataSource() {
	DriverManagerDataSource dataSource = new DriverManagerDataSource();
	dataSource.setDriverClassName(env.getRequiredProperty("jdbc.driverClassName"));
	dataSource.setUrl(env.getRequiredProperty("jdbc.url"));
	dataSource.setUsername(env.getRequiredProperty("jdbc.username"));
	dataSource.setPassword(env.getRequiredProperty("jdbc.password"));
	return dataSource;
	}
	
	@Bean
	public JdbcTemplate jdbcTemplate(DataSource dataSource) {
	JdbcTemplate jdbcTemplate = new JdbcTemplate(dataSource);
	jdbcTemplate.setResultsMapCaseInsensitive(true);
	return jdbcTemplate;
	}
	
	public Properties additionalProreties() {
	Properties properties = new Properties();
	//        properties.setProperty("hibernate.show_sql", "true");
	return properties;
	}
	
	@Bean
	public EntityManagerFactory entityManagerFactory() {
	HibernateJpaVendorAdapter vendorAdapter = new HibernateJpaVendorAdapter();
	vendorAdapter.setGenerateDdl(true);
	
	LocalContainerEntityManagerFactoryBean factory = new LocalContainerEntityManagerFactoryBean();
	factory.setJpaVendorAdapter(vendorAdapter);
	factory.setPackagesToScan("com.kspt.pms");
	factory.setDataSource(dataSource());
	factory.setPersistenceUnitName("my-pms");
	factory.setPersistenceProviderClass(HibernatePersistenceProvider.class);
	factory.setJpaProperties(additionalProreties());
	factory.afterPropertiesSet();
	return factory.getObject();
	}
	
	@Bean
	public PlatformTransactionManager transactionManager() {
	JpaTransactionManager txManager = new JpaTransactionManager();
	txManager.setEntityManagerFactory(entityManagerFactory());
	return txManager;
	}
	\end{lstlisting}
	