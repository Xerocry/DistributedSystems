\section{Инструкция пользователя}
При переходе на главную страницу сайта будет выведена форма для авторизации в системе. Если у пользователя нет аккаунта, он может зарегистрироваться на этой же странице. После успешной авторизации в системе, пользователь будет перенаправлен на свою страницу пользователя.

\subsection{Страница пользователя}
Внешний вид страницы пользователя приведен на рисунке~\ref{fig:userPage}. В верхней части страницы находится панель навигации, с помощью которой можно перейти на домашнюю страницу~(страница авторизованного пользователя) или выйти из системы. 

Под панелью навигации выводится общая информация о пользователе: идентификатор, логин, имя. Далее на данной странице предоставляется следующая информация:
\begin{itemize}
	\item список проектов, в которых данный пользователь является менеджером;
	\item список проектов, в которых данный пользователь является тимлидером;
	\item список проектов, в которых данный пользователь является разработчиков;
	\item список проектов, в которых данный пользователь является тестировщиком;
	\item список ошибок, которые были созданы данным пользователем;
	\item список ошибок, которые исправляются данным пользователем;
	\item список тикетов, которыми руководит данный пользователь;
	\item список тикетов, которые выполняет данный пользователь;
	\item список уведомлений, полученных данным пользователем.
\end{itemize}

Все приведенные в списке элементы~(за исключением уведомлений) являются активными, и при нажатии на них пользователь будет перенаправлен на страницу соответствующего объекта.

Если пользователь находится на своей домашней странице~(т.е. на странице пользователя, за которого он авторизован), ему также будет доступна возможность создания нового проекта. Для этого в соответствующее текстовое поле необходимо ввести имя нового проекта и нажать на кнопку \texttt{Create project}. В случае успеха список проектов, в которых данный пользователь является менеджером, обновится. Иначе появится информационное сообщение с описанием ошибки.

\subsection{Страница проекта}
Внешний вид страницы проекта приведен на рисунке~\ref{fig:projectPage}. В верхней части страницы находится панель навигации, с помощью которой можно перейти на домашнюю страницу~(страница авторизованного пользователя) или выйти из системы. 

Под панелью навигации выводится общая информация о проекте: идентификатор, имя, менеджер, тимлидер.

Для того, чтобы сделать пользователя тимлидером данного проекта, необходимо ввести логин пользователя в соответствующее поле и нажать на кнопку \texttt{Set team leader}. В случае успеха информация о проекте обновится. Иначе выведется информационное сообщение с описанием ошибки.

Далее на странице показано две колонки:
\begin{itemize}
	\item Первая колонка показывает пользователей, привязанных к проекту. Для того, чтобы добавить нового пользователя в проект необходимо ввести его логин в соответствующее поле и нажать на кнопку добавления. Если операция выполнится успешно, информация на странице обновится. Иначе выведется информационное сообщение с описанием ошибки.
	
	\item Во второй колонке показаны составляющие проекта: майлстоуны и отчеты об ошибках. Для того, чтобы добавить новый майлстоун необходимо выбрать в соответствующих полях даты его начала и завершения и нажать на кнопку \texttt{Add milestone}. Если операция выполнится успешно, информация на странице обновится. Иначе выведется информационное сообщение с описанием ошибки.
	
	Для того, чтобы добавить новый отчет об ошибке, необходимо ввести в соответствующее текстовое поле описание ошибки и нажать на кнопку \texttt{Add report}. Если операция выполнится успешно, информация на странице обновится. Иначе выведется информационное сообщение с описанием ошибки.
\end{itemize}

Все элементы на странице являются активными, т.е. при нажатии на них пользователь будет перенаправлен на соответствующую данному объекту страницу. 

Элементы управления~(кнопки добавления пользователей и т.д.) отображаются только в том случае, если у пользователя есть соответствующие права.

\subsection{Страница майлстоуна}
Внешний вид страницы майлстоуна приведен на рисунке~\ref{fig:milestonePage}. В верхней части страницы находится панель навигации, с помощью которой можно перейти на домашнюю страницу~(страница авторизованного пользователя) или выйти из системы. 

Под панелью навигации выводится общая информация о майлстоуне: идентификатор, проект, предполагаемая дата начала, фактическая дата начала~(если она есть), предполагаемая дата завершения, фактическая дата завершения~(если она есть).

Под общей информацией находятся кнопки управления майлстоуном: кнопка активации майлстоуна и кнопка закрытия майлстоуна. Они отображаются пользователю только в том случае, если у него есть права на управление майлстоуном. Для того, чтобы поменять статус майлстоуна необходимо нажать на соответствующую кнопку. Если операция выполнится успешно, информация на странице обновится. Иначе выведется информационное сообщение с описанием ошибки.

Ниже на странице отображается список всех тикетов, привязанных к данному майлстоуну. При нажатии на тикет пользователь будет перенаправлен на страницу данного тикета.

Для того, чтобы добавить новый тикет к майлстоуну необходимо ввести описание задачи в соответствующее поле и нажать кнопку \texttt{Create ticket}. Если операция выполнится успешно, информация на странице обновится. Иначе выведется информационное сообщение с описанием ошибки.

\subsection{Страница тикета}
Внешний вид страницы тикета приведен на рисунке~\ref{fig:ticketPage}. В верхней части страницы находится панель навигации, с помощью которой можно перейти на домашнюю страницу~(страница авторизованного пользователя) или выйти из системы. 

Под панелью навигации выводится общая информация о тикете: идентификатор, майлстоун, дата создания, создатель, статус, описание задачи.

Под общей информацией располагаются кнопки управления статусом тикета. Они доступны пользователю только в том случае, если он обладает какими-либо правами в данном проекте. Для того, чтобы изменить статус тикета, необходимо нажать соответствующую кнопку. Если операция выполнится успешно, информация на странице обновится. Иначе выведется информационное сообщение с описанием ошибки.

Ниже на странице показывается две колонки:
\begin{itemize}
	\item Список разработчиков тикета. Для того, чтобы добавить нового разработчика к тикету, необходимо ввести его логин в соответствующее поле и нажать на кнопку \texttt{Add assignee}. Если операция выполнится успешно, список разработчиков обновится. Иначе выведется информационное сообщение с описанием ошибки. Данные кнопки доступны пользователю только при наличии соответствующих прав.
	
	\item Список комментариев к тикету. Для того, чтобы добавить новый комментарий к тикету, необходимо ввести содержание комментария в соответствующее поле и нажать на кнопку \texttt{comment}. В случае успеха в списке комментариев появится добавленный комментарий. Иначе выведется информационное сообщение с описанием ошибки. Данные кнопки отображаются пользователю только в том случае, если он обладает правами на комментирование тикета.
\end{itemize}

\subsection{Странице отчета об ошибке}
Внешний вид страницы отчета приведен на рисунке~\ref{fig:reportPage}. В верхней части страницы находится панель навигации, с помощью которой можно перейти на домашнюю страницу~(страница авторизованного пользователя) или выйти из системы. 

Под панелью навигации выводится общая информация об отчете: идентификатор, проект, дата создания, создатель, статус, описание задачи, разработчик~(если назначен).

Далее на странице показывается список комментариев к отчету. Для того, чтобы добавить новый комментарий к отчету, необходимо ввести содержание комментария в соответствующее поле и нажать на кнопку \texttt{comment}. В случае успеха в списке комментариев появится добавленный комментарий. Иначе выведется информационное сообщение с описанием ошибки. Данные кнопки отображаются пользователю только в том случае, если он обладает правами на комментирование отчета.

