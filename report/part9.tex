\section{Выводы}
	В рамках данной работы были изучены принципы работы с ORM Hibernate, принципы работы с технологиями Spring Data и Spring MVC, создание распределенных веб-приложений на языке Java. Также были изучены основы создания RESTful-клиентов с использованием фреймворка AngularJS. Поставленные в рамках работы задачи были выполнены. Однако, полученное приложение можно далее развивать в нескольких направлениях:
	\begin{itemize}
		\item улучшение интерфейса;
		\item расширение функциональности текущих ролей;
		\item добавление новых ролей и вариантов использования.
	\end{itemize}
	
	Использование библиотек ORM и Spring Data позволило значительно облегчить разработку слоя источников данных. ORM позволяет сделать многие вещи, связанные с хранением данных, прозрачными для программиста, а Spring Data дает возможность автоматической генерации всех необходимых методов обращения к слою хранения. Благодаря архитектуре созданного приложения, оно должно быть легко масштабируемым, однако данное свойство не было проверено в рамках курсовой работы.
	