\section{Роли}
В системе определены следующие роли для пользователей:
\begin{itemize}
	\item менеджер;
	\item тимлидер;
	\item разработчик;
	\item тестировщик.
\end{itemize}

Для каждого проекта у пользователя определена своя роль~(если он участвует в разработке данного проекта).

У всех пользователей системы есть возможность:
\begin{itemize}
	\item зарегистрироваться;
	\item просмотреть все проекты в которых они участвуют;
	\item посмотреть список заданий, который был им выдан;
	\item посмотреть список отчетов об ошибках, которые ему надо исправить;
	\item создать новый проект.
\end{itemize}

Функции менеджера проекта:
\begin{itemize}
	\item Управление пользователями:
	\begin{itemize}
		\item назначение тимлидера
		\item добавление разработчика к проекту
		\item добавление тестировщика к проекту
	\end{itemize}
	
	\item Управление майлстоунами
	\begin{itemize}
		\item создание нового майлстоуна
		\item изменение статуса майлстоуна
	\end{itemize}
	
	\item Управление тикетами
	\begin{itemize}
		\item создание нового тикета
		\item привязка разработчика к тикету
		\item проверка выполнения тикета
	\end{itemize}
\end{itemize}

Функции тимлидера:
\begin{itemize}
	\item Управление тикетами
	\begin{itemize}
		\item создание нового тикета
		\item привязка разработчика к тикету
		\item проверка выполнения тикета
	\end{itemize}
	
	\item Выполнение тикетов
\end{itemize}

Функции разработчика:
\begin{itemize}
	\item Выполнение тикетов
	\item Создание сообщений об ошибках
	\item Исправление сообщений об ошибках
\end{itemize}

Функции тестировщика:
\begin{itemize}
	\item Тестирование проекта
	\item Создание сообщений об ошибках
	\item проверка исправления сообщений об ошибках
\end{itemize}